\fixmargins
\makepreliminarypages
\oneandhalfspace
\pdfbookmark[section]{\contentsname}{toc}
\tableofcontents

\section{Введение}

Традиционным видом практических заданий по IT-дисциплинам являются задачи
на разработку программ, связанных с изучаемыми технологиями, библиотеками
и алгоритмами. Набор подобных задач по дисциплине «Операционные системы» 
приводится, например, в пособии \cite{ref16} и в завершающих разделах задачника \cite{ref17}. 
Как сам процесс решения задач студентами, так и проверку решенных задач 
преподавателем можно упростить и ускорить, если при решении и проверке 
задач использовать специализированные программные средства — электронные 
задачники \cite{ref18}.

Примером электронного задачника по программированию является задачник 
Programming Taskbook. Хотя особенности задачника Programming Taskbook не 
позволяют разработать его расширение для курса по операционным системам 
(поскольку текущая версия задачника реализована только для ОС Windows), 
принципы, лежа- щие в основе его архитектуры \cite{ref19}, были использованы в задачника 
Unix Taskbook.

Задачник Unix Taskbook реализован на языке C++ для Unix и включает ядро, 
обеспечивающее его основную функциональность, и набор интерфейсов, которые 
позволяют нам легко расширить целевую группу Unix Taskbook.Эта диссертация 
использует этот набор интерфейсов для расширения динамической библиотеки групп 
задач по параллельной программированию для Unix Taskbook.

В дополнение к этим динамическим библиотекам, также был разработан специальный 
модуль, подключаемый к параллельному приложению и позволяющий каждому процессу 
считывать исходные данные, подготовленные задачником, и пересылать результаты 
на проверку. Прежде всего работа дается подробное описание особенностей архитектур 
задачников Programming Taskbook и Unix Taskbook. Во-вторых посвящен дополнительный
 модуль ut1.cpp, подключаемый к учебной программе. И потом, описывает реализацию 
 набора динамических библиотек, подключаемых к задачнику Unix Taskbook. Наконец, 
 показано практическое применение данного набора динамических библиотек.


