\fixmargins
\makepreliminarypages
\oneandhalfspace
\pdfbookmark[section]{\contentsname}{toc}
\tableofcontents

\section{Введение}

Традиционным способом создания практических заданий в области IT-дисциплин 
является разработка программ, связанных с изучаемыми технологиями, библиотеками 
и алгоритмами. Использование специализированных программных средств - электронных задачников \cite{ref18} 
может значительно упростить и ускорить процесс решения и проверки задач как студентами, так и преподавателями.

Примером электронного задачника по программированию является Programming Taskbook. 
Хотя уже существуют группы задач по параллельному программированию в задачнике Programming 
Taskbook, в настоящее время он работает только на системах Windows.  
Поэтому, чтобы создать версию, которая будет работать под Unix-системами, была создана unixTaskbook\cite{ref1}.
Разработка unixTaskbook основана на принципы, лежащие в основе архитектуры задачника Programming Taskbook\cite{ref19}.
Затем поверх unixTaskbook можно легко перенести группы задач из задачника Programming Taskbook.


Задачник Unix Taskbook реализован на языке C++ для Unix и включает ядро, 
обеспечивающее его основную функциональность, и набор интерфейсов, которые 
позволяют нам легко расширить целевую группу Unix Taskbook.Эта диссертация 
использует этот набор интерфейсов для расширения динамической библиотеки групп 
задач по параллельной программированию для Unix Taskbook.

В дополнение к этим динамическим библиотекам, также был разработан специальный 
модуль, подключаемый к параллельному приложению и позволяющий каждому процессу 
считывать исходные данные, подготовленные задачником, и пересылать результаты 
на проверку. Прежде всего работа дается подробное описание особенностей архитектур 
задачников Programming Taskbook и Unix Taskbook. Во-вторых посвящен дополнительный
 модуль ut1.cpp, подключаемый к учебной программе. И потом, описывает реализацию 
 набора динамических библиотек, подключаемых к задачнику Unix Taskbook. Наконец, 
 показано практическое применение данного набора динамических библиотек.


