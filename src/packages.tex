\usepackage{cmap} % Для корректной кодировки в pdf
\usepackage[utf8]{inputenc}
\usepackage{rotating}
\usepackage[russian]{babel}
\usepackage{amsfonts} % Пакеты для математических символов и теорем
\usepackage{amstext}
\usepackage{amssymb}
\usepackage{amsthm}
\usepackage{graphicx} % Пакеты для вставки графики
\usepackage{subfig}
\usepackage{color}
\usepackage[unicode]{hyperref}
\usepackage[nottoc]{tocbibind} % Для того, чтобы список литературы отображался в оглавлении
\usepackage{verbatim} % Для вставок заранее подготовленного текста в режиме as-is
\usepackage{listings}

\newcommand{\sectionbreak}{\clearpage} % Раздел с новой станицы

\usepackage{tikz}
\usepackage{pgfplots}
\usetikzlibrary{arrows,positioning}
\usepackage{adjustbox}
\usepackage{makecell}
\usepackage{booktabs}
\usepackage{boldline}
\usepackage{xcolor}
\usepackage{soul}
\usepackage{url}
\usepackage{multirow}
\usepackage{amsmath}
\usepackage{pifont}
\usepackage{indentfirst} % Делать отступ в начале первого параграфа

\usepackage{listings}
\usepackage{xcolor}
% \lstset{
%   language=C++,
%   basicstyle=\footnotesize,
%   numbers=left,
%   numberstyle=\footnotesize,
%   keywordstyle=\color{blue},
%   commentstyle=\color{green},
%   frame=single,
%   breaklines=true
% }
\lstset{
    numbers = left,
    numberstyle = \tiny,
    keywordstyle = \color{ blue!70},
    commentstyle = \color{red!50!green!50!blue!50},
    frame = shadowbox,
    rulesepcolor = \color{red!20!green!20!blue!20},
    xleftmargin = 2em, xrightmargin = 2em, aboveskip = 1em,
    framexleftmargin = 2em
}

% Общие параметры листингов
\lstset{
    %frame=TB,
    showstringspaces=false,
    tabsize=4,
    basicstyle=\linespread{1.0}\tt\small, % делаем листинги компактнее
    breaklines=true,
    texcl=true, % русские буквы в комментах
    captionpos=b,
    aboveskip=\baselineskip,
    commentstyle=\tt
}
\newcommand{\todo}[1]{\textcolor{red}{#1}}