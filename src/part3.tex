\section{Модуль, который отвечает за взаимодействие с базовой динамической библиотекой и проверкой решения}
Базовая динамическая библиотека содержит весь текст и данные, необходимые для генерации задач.

Поскольку наша Unix Taskbook адаптирована из Programming Taskbook, удобно использовать базовые динамические библиотеки, используемые в Programming Taskbook напрямую, так что нам не нужно разрабатывать отдельный набор базовых динамических библиотек для UnixTaskbook.

Для того чтобы установить контакт с базовой динамической библиотекой, мы разработали интерактивный модуль p t4utilities, который в основном используется для загрузки базовой динамической библиотеки и проверки решения.


\subsection{Функции, которые отвечают за взаимодействие с базовой динамической библиотекой}
\lstset{language=c++}
\begin{lstlisting}
bool load_lib()
{
    bool result = false;
    free_lib();
#if defined __linux__
    dylib_name += ".so";
#elif defined __APPLE__
    dylib_name += ".dylib";
#endif
    FLibHandle = dlopen(dylib_name.c_str(), RTLD_LAZY);
    if (FLibHandle == nullptr)
    {
        return false;
    }
    InitGroup_ = (TInitGroup)dlsym(FLibHandle, "initgroup");
    if (InitGroup_ == nullptr)
    {
        free_lib();
        return false;
    }
    InitTask_ = (TInitTask)dlsym(FLibHandle, "inittask");
    if (InitTask_ == nullptr)
    {
        free_lib();
        return false;
    }
    return true;
}
\end{lstlisting}

\textbf{load\_lib} - это функция, которая загружает базовую динамическую библиотеку. Сначала она определяет тип операционной системы, на которой запущена программа, а затем загружает динамическую библиотеку с суффиксом динамической библиотеки, соответствующей этой системе. Затем она получает функции \textbf{initgroup} и \textbf{inittask}, предоставляемые базовой динамической библиотекой.

Две другие функции в pt4utilities, \textbf{pt4\_print\_task\_info} и \textbf{pt4\_generate\_task\_test}, используют интерфейсы, предоставляемые initgroup и inittask.

\subsection{Функции, которые отвечают за проверкой решения}
\lstset{language=c++}
\begin{lstlisting}
int pt4_check_program(std::string task_group, int test_num)
{
    bool result = false;
    result = CheckAllResults();
    return result ? 0 : 1;
}
\end{lstlisting}

Для проверки решения используется функция \textbf{pt4\_check\_program}, которая вызывает функцию \textbf{CheckAllResults} из другого модуля, который будет рассмотрен в следующей главе.